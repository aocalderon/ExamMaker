\documentclass[11pt, answers, addpoints]{exam}
\usepackage[utf8]{inputenc}
\usepackage[spanish]{babel} % Para idioma español y separación silábica
\usepackage[many]{tcolorbox} % for COLORED BOXES (tikz and xcolor included)
\newtcolorbox{MyBox}{
	fontupper = \bf \color{black}, % font color
    boxrule = 2pt,
    colframe = black,
    rounded corners,
    arc = 12pt
}
\usepackage{qrcode}
\usepackage{amsmath}
\usepackage{graphicx}
\usepackage{amssymb}
\usepackage{geometry}
\geometry{margin=2cm, bottom=3cm}

\begin{document}
	\input{../code128}
	\begin{coverpages}
		\begin{center}
			\begin{Large}
				\vspace{-5mm}
				ADA Exam
			\end{Large}
		\end{center}
		\vspace{5mm}

		\begin{MyBox}
			This exam consists of \numpages{} pages, not including this cover page.  Please go through your copy to make sure that all pages are in good order.  The exam consists of a set of short questions with multiple choices. There are 20 points, total, on this exam. \\
			Happy solving!
		\end{MyBox}

		\vspace{10mm}
		\noindent
		Name:\enspace\hrulefill \\
		\\
		Date:\enspace\hrulefill
		\vspace{5mm}
		\begin{center}
			\includegraphics{../answer_table}

			\vspace{3mm}
			\leavevmode \hspace{5mm} \codetext{1AF9D0}
		\end{center}
	\end{coverpages}

	\footer{} {Page \thepage\ of \numpages} {\iflastpage{3rd exam.}{Please go on to the next page\ldots}}

	\centering
	\textbf{\Large Pontificia Universidad Javeriana}\\
	\textbf{\Large ADA} \\
	\textbf{\large 2025 -- 10} \\
	\textbf{\large 3rd Exam} \\
	\textbf{Code: 1AF9D0}


\begin{questions}

\question ¿Cuál es una característica clave de los algoritmos de fuerza bruta?
\begin{choices}
\choice Realizan una búsqueda parcial optimizada.
\CorrectChoice Evalúan todas las posibles soluciones sin podarlas.
\choice Usan estructura de divide y vencerás para mejorar el rendimiento.
\choice Solo funcionan con entradas grandes y datos ordenados.
\end{choices}

\question ¿Qué problema se resuelve comúnmente con fuerza bruta?
\begin{choices}
\choice Búsqueda binaria.
\CorrectChoice Problema del Viajante (TSP).
\choice Árbol de expansión mínima.
\choice QuickSort.
\end{choices}

\question ¿Qué propiedad es esencial para aplicar programación dinámica?
\begin{choices}
\choice Aleatoriedad en los datos.
\CorrectChoice Subproblemas superpuestos.
\choice Datos en forma de árbol.
\choice Entrada ya ordenada.
\end{choices}

\question ¿Cuál es la finalidad principal de usar memoización?
\begin{choices}
\choice Reducir uso de memoria.
\CorrectChoice Evitar cálculos redundantes almacenando resultados.
\choice Generar pseudocódigo automáticamente.
\choice Aumentar precisión numérica.
\end{choices}

\question ¿Qué estrategia se usa en divide y vencerás?
\begin{choices}
\choice Eliminar duplicados antes de resolver.
\CorrectChoice Dividir el problema en subproblemas, resolverlos y combinar.
\choice Iterar hasta que el input sea vacío.
\choice Ordenar la entrada antes de iniciar.
\end{choices}

\question ¿Cuál algoritmo es un ejemplo clásico de divide y vencerás?
\begin{choices}
\choice DFS.
\choice Fuerza bruta para TSP.
\CorrectChoice MergeSort.
\choice Knapsack.
\end{choices}

\question ¿Qué propiedad garantiza que un algoritmo greedy sea óptimo?
\begin{choices}
\choice Complejidad logarítmica.
\CorrectChoice Subestructura óptima.
\choice Entrada ordenada.
\choice Uso de árboles binarios.
\end{choices}

\question ¿Cuál de los siguientes problemas se puede resolver con programación dinámica?
\begin{choices}
\choice Dijkstra.
\CorrectChoice Subset Sum.
\choice Búsqueda binaria.
\choice TSP con DFS.
\end{choices}

\question ¿Qué significa la notación \(\mathcal{O}(n^2)\)?
\begin{choices}
\CorrectChoice Cota superior del tiempo de ejecución en el peor caso.
\choice Cota inferior del tiempo de ejecución.
\choice Tiempo exacto que toma el algoritmo.
\choice Espacio de memoria utilizado.
\end{choices}

\question ¿Qué técnica se usa para verificar la corrección de un algoritmo?
\begin{choices}
\choice Analizar el uso de memoria.
\CorrectChoice Comprobar precondiciones y postcondiciones.
\choice Reducir el tamaño de la entrada.
\choice Usar programación orientada a objetos.
\end{choices}

\question ¿Qué implica una estrategia de poda en backtracking?
\begin{choices}
\CorrectChoice Eliminar caminos que no pueden llevar a soluciones válidas.
\choice Dividir el input en mitades iguales.
\choice Usar estructuras de árbol AVL.
\choice Ejecutar el algoritmo más de una vez.
\end{choices}

\question ¿Cuál es una ventaja de usar pseudocódigo en el diseño de algoritmos?
\begin{choices}
\choice Ejecuta más rápido que el código real.
\CorrectChoice Mejora la claridad y legibilidad del algoritmo.
\choice Optimiza automáticamente el código.
\choice Genera pruebas unitarias.
\end{choices}

\question ¿Qué herramienta ayuda a identificar cuellos de botella en algoritmos?
\begin{choices}
\choice Visual Studio Code.
\CorrectChoice Perfiladores como \texttt{cProfile} o \texttt{gprof}.
\choice Compiladores.
\choice Exploradores de archivos.
\end{choices}

\question ¿Qué notación representa la cota ajustada del tiempo de ejecución?
\begin{choices}
\choice \(\Omega(n)\)
\CorrectChoice \(\Theta(n)\)
\choice \(\mathcal{O}(n)\)
\choice \(\delta(n)\)
\end{choices}

\question ¿Qué algoritmo puede beneficiarse del uso de bitmasking?
\begin{choices}
\choice MergeSort.
\CorrectChoice Solución de Sudoku por backtracking.
\choice BFS en árboles.
\choice Dijkstra.
\end{choices}

\question ¿Qué estrategia mejora la eficiencia de QuickSort?
\begin{choices}
\choice Reemplazar por fuerza bruta en todos los casos.
\CorrectChoice Usar selección aleatoria del pivote.
\choice Repetir partición tres veces.
\choice Usar estructuras dinámicas.
\end{choices}

\question ¿Qué significa una complejidad de \(\mathcal{O}(1)\)?
\begin{choices}
\choice El algoritmo no termina nunca.
\CorrectChoice Tiempo constante, independiente del tamaño de entrada.
\choice Tiempo cuadrático.
\choice Tiempo exponencial.
\end{choices}

\question ¿Cuál es un ejemplo de estructura útil en programación dinámica?
\begin{choices}
\choice Árbol binario de búsqueda.
\CorrectChoice Tabla de memoización.
\choice Grafo dirigido acíclico.
\choice Árbol AVL.
\end{choices}

\question ¿Cuál técnica implica recorrer manualmente un algoritmo con ejemplos concretos?
\begin{choices}
\choice Refactorización.
\CorrectChoice Ejecución en seco (dry run).
\choice Compilación anticipada.
\choice Optimización recursiva.
\end{choices}

\question ¿Qué algoritmo es un ejemplo clásico de greedy?
\begin{choices}
\choice MergeSort.
\CorrectChoice Codificación de Huffman.
\choice Subset Sum con memoización.
\choice Floyd-Warshall.
\end{choices}

\end{questions}


	\vspace{5mm}
	\noindent \textbf{End of Exam}

\end{document}
